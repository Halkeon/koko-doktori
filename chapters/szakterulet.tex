\chapter{A szakterület áttekintése}

Az anyagok törésének és roncsolódásának vizsgálata egy érdekes és fontos tudományos és technikai, mérnöki probléma. Ennek elméleti vizsgálatára mind a mérnöki mind a fizikus közösség széles körben alkalmazza az úgynevezett szálköteg modellt.
A modellt először Peirce (Peires) használta 1926-ban megjelent úttörő munkájában. Pierce egy meglehetősen gyakorlati probléma vizsgálatával jutott el a modell egy korai változatának megalkotásához. Munkájában egy alapos, minden részletre kiterjedő kísérletsorozatot végzett a gyapot ahol egy meglehetősen gyakorlati probléma a gyapot szálak kötegeinek teherbírását, szakítószilárdságát vizsgálta ezzel a módszerrel




\section{Monte Carlo szimulációk}

\section{Sejtautomaták}

\section{Eseményvezérelt szimulációk}

\section{Statisztikus fizika rövid áttekintése}

