\chapter{Crackling noise in non-destructive material testing}

Nagy mechanikai terhelésnek kitett anyagokat gyakran készítenek úgy, hogy egy viszonylag gyenge mátrixanyagot erős szálakkal erősítenek meg. 
Az ilyen módon megerősített anyagok konstans, vagy lassan növekvő terhelés hatására egy fokozatos mikro-törés felhalmózódási, károsodási folyamaton mennek keresztül, ami végül
az anyag, köznapi szóval elfáradásához, tönkremeneteléhez vezet.Mivel ilyen terhelés és ezáltal ilyen károsodás jellemzően, bár nem kizárólag valamilyen tartószerkezetben, van hosszú távra tervezett struktúrában fordul elő, ezért a biztonságos használat érdekében az ilyen anyagokat, különösen a koruk előrehaladtával érdemes folyamatos megfigyelésnek, vizsgálatoknak alávetni. 
A károsodási folyamatot a kibocsátott akusztikus jelek folyamatos figyelésével követhetjük. Ezt a "zajt" a repedések képződése kelti az anyagban és a jel nagysága, hangossága a repedés által eldisszipált energiával arányos. Ez az úgynevezett 