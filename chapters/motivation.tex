\chapter{Motiváció}
\label{chapt:motivation}

A dolgozat témáját tekintve a számítógépes és statisztikus fizika területére kalauzol minket, erősen támaszkodva a számítástechnika, informatika és numerikus módszerek eredményeire.



Az egyes

A számítógépes fizika egy teljesen úgy világot nyitott meg a tudomány előtt, mivel addig anyagi vagy technikai korlátok miatt nem vizsgálható folyamatok váltak vizsgálhatóvá. A kontrollált kísérletezés előtt is határtalan lehetőségeket nyitott a számítógépes fizika, hiszen a labor kisérletektől eltérően is tényleg, magunk vagyunk a világ irányítói és minden paraméter és minden törvény a mi elképzelésein szerint alakul, eltekintve azoktók amelyek Murphy törvényeit követik. 


Motivációm....	


Mind a fragmentáció mind a szálkötegmodell jelentős történelmi háttérrel rendelkezik. 

Mint ahogy az elvégzett munka is kettős ugyanúgy a motiváció is. A fragmentációs folyamatok mélyebb megértése rávilágíthat sok sok mindenre, fogalmam sincs hogy mikre, de nagyon nagyon sok mindenre tényleg. A motivációm pedig az volt, hogy egy dr. legyen a nevem előtt. Ha ez nem lenne elegendő, akkor még azt is hozzáteszem, hogy .....

Hajjaj

A dolgozat két módszertanilag különböző részre osztható. 
a következőket szeretném mondani.... hogy   

csak na

bizony

A dolgozat a szilárd testek törésének számítógépes szimulációjáról szól. Az évek során elvégzett munka alapvetően két jól elkülöníthető részre tagolható a használt modellek, technikák alapján. Ez a tagolás látszólag megfelel az "elméletibb" illetve a "gyakorlatibb" szerinti tagolásnak is, de mint arra a későbbiekben példákat láthatunk, az egészen egyszerű elméleti analitikusan is kezelhető modellek is meglepően jól jellemeznek kvalitatíve valós folyamatokat, és habár az egyezés nem tökéletes de még a kvantitatív eredmények összevetése sem teljesen elrugaszkodott gondolat. 

A két modell amellyel dolgoztam, a szálköteg modell (Fiber Bundle Model - FBM), és egy a szemcsés anyagok vizsgálatát molekuláris dinamikai technikával vizsgáló modell. Az FBM ismereteim szerint részben technika részben modell, míg az MD egy széles körben alkalmazott technika amelyet vegyészek, biológusok és játélfejlesztők is széles körben használnak. Annak, hogy mit vizsgálunk vele csak képzeletünk, a rendelkezésünkre álló számítókapacitás és saját elméleti tudásunk szabhat határt. Fontos már a bevezetőben hangsúlyoznom, hogy minden számítógép szimuláción alapuló munka megköveteli a kutatótól, hogy nagyon óvatosan járjon el, nehogy abba a hibába essen, hogy tudomány helyett számítógépes játékot készít.
Mivel ma már a kísérletek is a technikai teljesíthetőség határát feszegetik ezért akár egy kísérletből is kaphatunk olyan "új fizikát" ami jelenlegi tudásunk szerint nem létezhet, ha nem vagyunk elég óvatosak - lásd a fénynél gyorsabb részecske felfedezését HIVATKOZÁS - azonban ilyen eredményeket a számítógép segítségével még könnyebben állíthatunk elő.


. 


Közismert tény, hogy a dolgok eltörnek. Arról lehet vitát nyitni, hogy mekkora erőhatásra van mindehhez szükség, de az biztos hogy ha a rombolás energiája kellően nagy akkor bizony kő kövön nem marad. Azt sem vitatja senki, hogy szinte összehasonlíthatatlanul (ez természetesen költői túlzás) több energiára van szükség egy tank mint egy váza eltöréséhez, már amennyiben a tank teljesít bizonyos minőségi elvárásokat. Valószínű hogy az energia közlés módja is jelentősen különbözik. A tankot talán nagyon magasról ejtettük le, míg a vázát aknák segítségével robbantottuk fel. Esetleg fordítva.

Tételezzük fel, hogy mindkét esetben elegendően nagy az energia ahhoz, hogy a céltárgyunk ne csak néhány - például a 20 ujjunk segítségével megszámlálható - darabra törjön, hanem annál jóval többre - tehát több mint 20. Ezt a folyamatot a szakirodalom fragmentációnak nevezi. Maga a fragmentáció mára jelentősen túlterhelt fogalommá vált, azonban minden eseteben valamilyen korábban egy egységet alkotó entitás több kisebb, nagyobb darabra törését jelenti. 

Érdekes kérdés, hogy ez a két, hétköznapi vélekedés szerint teljesen különböző folyamat mutat-e valamilyen kvantitatív hasonlóságot. A dolgozat első felében egy konkrét fragmentációs kísérlet ellentmondó eredményeinek szimulációs vizsgálatáról lesz szó.

A dolgozat második felében egy igen egyszerű elméleti modellt a szálkötegmodellt veszem górcső alá. Doktori munkám részeként a modell egy kiegészítését és annak következményeit vizsgáltam meg. Míg a fragmentáció egy rövid idő alatt lezajló folyamat addig a szálkötegmodell keretében vizsgált törések általában az anyagok teherbírásának vizsgálatához járulnak hozzá. Ahol az adott anyagot statikus állapotokon keresztül terheljük és megvizsgáljuk hogy milyen feltételek mellett jut el addig a pontig amikor egy katasztrófikus esemény keretében az anyag eltörik. 

A két terület módszertanilag is különbözik. A fragmentációt a molekuláris dinamika eszközeivel fogom vizsgálni, a szálkötegmodellt egy rács modell segítségével. 




Doktori dolgozatom illetve doktori munkám első felében ehhez a tudományterülethez igyekeztem számítógépes szimuláció segítségével képességeimhez mérten hozzájárulni. 

Míg a fragmentációt általában nagy energia és  egy gyors törési folyamattal foglalkozik addig a dolgozat második felében A dolgozat másik felében is anyagok törésének számítógépes vizsgálatáról lesz szó, de ez szorosabban kapcsolódik az anyagok terhelhetőségének, azon belül az erősen rendezetlen anyagok terhelés hatására  vizsgálatához.




Az állatvilágban sok példát találhatunk eszközhasználatra. A legtöbb esetben ez valamilyen a természet által alkotott eszköz használatát jelenti. Valószínűleg korai őseink sem sokat törődtek azzal ha az ásásra használt egyszerű bot eltört a kezükben, eldobták és kerestek másikat megvalósítva ezzel a tökéletes fogyasztói magatartást. Abban viszont biztos vagyok, hogy sokkal érzékenyebben érintette őket amikor a saját találékonyságuk eredményeként saját kezűleg elkészített eszközeik tönkrementek. Azt nem tudhatjuk, hogy pontosan esetleg valamilyen szellemnek vagy misztikus oknak tulajdonították ezt a balszerencsét, de abban biztosak lehetünk, hogy idővel valakiben vagy valakikben felmerült, hogy talán mégiscsak valamilyen megérthető ok áll annak hátterében, hogy a tárgyak elhasználódnak és ez ellen valószínűleg tenni is lehet. 

Valószínűleg a legelső ember által készített eszköz elkészítése fölött érzett eufóriát 

A számítógépek megjelenésével új távlatok nyíltak a tudomány előtt. Az 1920-as évekig a számításokat kalkulusban képzett adminisztrátorok végezték.  
Először csupán az adattárolást és adatfeldolgozást tették hatékonyabbá azonban a teljesítmény növekedésével új tudományterületek megszületéséhez vezettek. 

A hétköznapi ember a törés tulajdonságai mellett

