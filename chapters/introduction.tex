\chapter*{Bevezetés}
\label{chapt:introduction}

Az informatika és számítástechnika meglátásom szerint ma az a határtudomány amely a matematika egy évszázaddal ezelőtti szerepét igyekszik betölteni, abban a tekintetben, hogy olyan eszközöket szolgáltat a többi tudomány számára, amellyel addig le nem írható, meg nem vizsgálható területekre kalauzolja a kutatókat. 
Természetesen mint a matematikának a számítástudománynak és az informatikának is megvannak a szigorúan a saját területét érintő kutatásai és eredményei, de jelentős mértékben változtatta meg a kutatás módszertanát szinte az össze tudományterületen. %és akkor ezzel most mit akartál mondani? 
A doktori munkám során én az informatika, a számítástudomány, a számítógépes fizika, a statisztikus fizika és a klasszikus fizika eredményeire alapozva végeztem a kutatásaimat.
A következő fejezetekben a teljesség igénye nélkül áttekintem a dolgozatom témájához kapcsolódó fontosabb eredményeket, és elméleti tudnivalókat. A motiváció és elméleti bevezetés után két nagy fejezeteben tekintem át a doktori munka eredményeit. A munkám során a szilárd testek törésének két mind szimulációs technika mind modell mind releváns időskála, energiaközlés szempontjából különböző megközelítésével foglalkoztam.
A fragmentációs vizsgálatok 


A következő fejezetekben a teljesség igénye nélkül áttekintem a dolgozatom témájához kapcsolódó fontosabb eredményeket, és elméleti tudnivalókat. A motiváció és elméleti bevezetés után két nagy fejezeteben tekintem át a doktori munka eredményeit. A munkám során a szilárd testek törésének két mind szimulációs technika mind modell mind releváns időskála, energiaközlés szempontjából különböző megközelítésével foglalkoztam.


% Az elejét még át kell majd írni az alapján, hogy mit találok a quantum fizika a relativitás elmélet matematikai hátterérőlUtánna olvasni gyorsan kicsit a logikában leírt dolgaimnak a matematikai problémákról, wiki quantum mechanika 
%

A dolgozat témáját tekintve a számítógépes és statisztikus fizika területére kalauzol minket, erősen támaszkodva a számítástechnika, informatika és numerikus módszerek eredményeire.

Még a diplomamunkám során ismerkedtem meg a számítógépes
A diplomamunkám során hallgatóként ismerkedtem meg a molekuláris dinamika segítségével szimulálható problémák e

A következő fejezetekben először ismertetem a munkám során használt és továbbfejlesztett elméleti eredményeket, modelleket és szimulációs technikákat. Az elméleti bevezető után a dolgozat két mind modell mind szimulációs technika szempontjából jól elkülöníthető részre 

A dolgozat elméleti bevezetés fejezetében a teljesség igénye nékül tárgyalom azokat a szimulációs és modellezési technikákat amelyekre és amelyek általunk továbbfejlesztett változatára a doktori munkám épül. Munkám során két mind modell, mind szimulációs technika szempontjából jól elkülönülő témával foglalkoztam. Mindkét fejezet struktúrája tartalmazza a motivációt, a modell leírást, esetleges technikai részteletek majd az eredményeket. 
A dolgozat végén összegzem a munka eredményeit.