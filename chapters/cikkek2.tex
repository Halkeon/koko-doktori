\chapter{Sajat Cikkek}
\section{Avalanche dynamics of fiber bundle models}
Physical Review E 80151108 (2009)
\subsection{Abstract}
Bemutatjuk a folyamatosan károsodó szálkötegmodell (continuous damage fiber bundle model) részletes analitikus és numerikus vizsgálatát. A lineárisan elasztikus szálak az időfejlődésük során több részleges törésen mennek keresztül, aminek hatására a szálak merevsége törésről törésre csökken. Megmutatjuk, hogy az így alkotott modell egy sor különböző mechanikai jelenséget képes reprodukálni. A vizsgálatok során azt találtuk, hogy a az anyag makroszkopikusan megjelenő rideg illetve képlékeny viselkedését jól jellemezhetjük és előjelezhetjük a mikroszkopikusan mérhető megfelelő tulajdonságú lavinaeloszlásokkal, amelyek 5/2 és 2 közötti csökkenő hatványfüggvényt követnek, ezzel különbözve az átlag-tér szálköteg modellek vizsgálatánál tapasztalt eredményektől. Levezetjük továbbá zárt analitikus alakban a CDFBM egy családjának a fázis diagramját, ami a lehetséges lavina méret eloszlások egy jelentős részének alakulását leírja. Az eredményeink segítségével egységes képet alkothatunk a szálköteg modellekben vizsgált törési lavinák statisztikai tulajdonságairól.
\subsection{Bevezetés - Modell leírás}
A heterogén anyagok törésének és károsodásának vizsgálata a tudomány egy igen érdekfeszítő és izgalmas területe, amely magában hordozza több gyakorlati alkalmazás ígéretét. $REF1-3$ 
Közismert, hogy amennyiben az anyagot állandó, vagy nagyon lassan növekedő folyamatos terhelésnek tesszük ki, akkor az anyag törési hullámok, úgynevezett lavinák formájában károsodik, ezt úgy kell elképzelni, hogy az egyes lokalizált törések egymással korreláló törés sorozatokat indítanak, amelyeket nyugalmi periódusok választanak el egymástól$REF4-15$. 
Az anyag károsodásának mikroszkopikus vizsgálatához egy az akusztikus emissziós technikán alapuló módszerrel juthatunk kísérleti adatokhoz. Ebben a módszerben kihasználjuk azt a tényt, hogy a törés által disszipált energia - a lavina "nagysága" - arányos a kísérleti berendezés által mérhető hullám amplitúdójával. A közelmúlt eredményei kimutatták, hogy a kísérleti berendezéssel mérhető akusztikus jelek energiaeloszlás és amplitúdó függvényei hatványfüggvény viselkedést mutatnak, ahol az exponenesek bizonyos megszorítások mellett univerzálisnak tekinthetők $REF4-13$.

\\ A problématerületet vizsgáló elméleti megközelítések között a szálköteg modell (FBM Fiber Bundle Model) különleges helyet foglal el. Ez a kiemelt szerep abban rejlik, hogy miközben képes megragadni a rendezetlen anyagok törésének legfőbb elemeit, aközben kellőképpen egyszerű ahhoz, hogy az analitikus számítások se legyenek reménytelenek.$REF16-31$.
Az FBM esetén az anyagot párhuzamos szálak kötegeként diszkretizáljuk, amelyeket a hosszukkal megegyező irányú külső terhelésnek teszünk ki. A szálakat rugalmassági jellemzői azonosak, azonban a teherbírásuk valamilyen sztochasztikus eloszlást követ. 

Az szálkötegre ható terhelést kvázisztatikus módon növeljük, ennek eredményeként előbb a gyengébb szálak törnek el, az általuk korábban tartott terhelést a köteg tovább viseli, azaz valamilyen mechanizmus alapján szétosztódik a még sértetlen szálakra. Ennek az eredményeként újabb és újabb szálak törhetnek el egészen addig, amíg ez a folyamat le nem áll, vagy az egész köteg el nem szakad. 

A korábbi vizsgálatok megmutatták, hogy egyenlő terhelés elosztás esetén (ELS - Equal Load Sharing) a lavinaméret eloszlások hatványfüggvény viselkedést követnek, a terhelés eloszlások széles családján érvényes univerzális 5/2 exponenssel $REF17-22$. Az is be lett bizonyítva analitikusan, hogy amennyiben csupán a makroszkopikus törés kritikus pontja környékén keletkező lavinákat vesszük figyelembe az eloszlások elkészítésekor, akkor az exponens egy alacsonyabb 3/2 értéket vesz fel. Az a tény, hogy az exponens megváltozik a kritikus pont közelében módot ad olyan technikák megalkotására, amelyek segítségével a közelgő végzetes esemény előjelezhető$REF22,25,26$.

A törések statisztikai tulajdonságainak elméleti vizsgálata szempontjából nagyon érdekes kihívás megérteni ennek a skálafüggetlen viselkedésnek a hátterét, felderíteni a háttérben húzódó rendezetlenség és törési mechanizmusok szerepét, továbbá megtalálni a lehetséges univerzalitási osztályokat amik a törési folyamatot jellemzik$REF21,22,28-31$.
Egy korábbi munkánkban megmutattok, hogy amennyiben a köteget két jelentősen különböző száltípus keverékeként állítjuk elő, akkor az irodalomból jól ismert $5/2$-es kitevő helyett egy alacsonyabb $9/4$-es értéket mérhetünk  $REF23$.

Ebben a cikkben a különböző törési módozatok hatását vizsgáljuk meg különös tekintettel arra, hogy milyen hatással vannak a megfigyelhető lavina statisztikákra. A vizsgálatok elvégzéséhez a közelmúltban bemutatott CDFBM (Continuous Damage Fiber Bundle Model-t) használjuk $REF32-34$. A hagyományos DFBM modellel ellentétben, ahol a szálak a terhelési küszöbük elérésekor eltörnek és többé nem vesznek részt a köteg teherviselésében, a CDFBM esetén amikor egy adott szálra jutó terhelés eléri az adott szál előre meghatározott teherbírását, vagy más néven törési küszöbét, nem törik el azonnal, hanem részleges törések sorozatán keresztül folyamatosan veszít a merevségéből, ezzel az anyag lineáris válaszát módosítva, mint azt a későbbiekben be is mutatjuk. 
Ezt a több lépcsős törési folyamatot két paraméterrel jellemezhetjük, egyrészt egy 1-nél kisebb arányszámmal ami azt mondja meg, hogy egy adott törés után az anyag a korábbi merevségének milyen hányadát tartja meg, továbbá a megengedett törések számával. Ennek a két paraméternek a megfelelő változtatásával a modell alkalmas arra, hogy azzal az anyag mechanikus terhelésre adott különböző válaszait modellezzük, egészen a tökéletesen képlékeny viselkedéstől a kvázirideg reakción keresztül a felkeményedésig. A továbbiakban megmutatjuk, hogy a törési-lavinák méreteloszlásának jól ismert hatványfüggvény viselkedése a makroszkópikus mechanikai választ leíró konstitutív (????? állapotfüggvény, Zoli dolgozatában utána nézni) függvénytől függően különböző $5/2$ és $2$ közötti exponensekkel jelemezhető. (?????) Ezeket az eredményeket analitikus számításokkal is igazoltuk, majd analitikus módszerekkel megalkottuka modell fázisterét, ami áttekintést nyújt a modell keretein belül mefigyelhető különböző lavinaméret eloszlásokról. (????? Zoli dolgozatában utána nézni, hogy konzisztens legyen), majd azokat számítógépes szimulációkkal ellenőriztük. 

\subsection{Continuous Damage Fiber Bundle model - Fokozatosan károsodó Szálkötegmodell}
A CDFBM a klasszikus FBM egy kiterjesztése $REF32-32$. Az alap modellhez hasonlóan ebben a kitejesztésben is $N$ egymással párhuzamos, azonos $E_f$ Youn-moduluszú szál alkotja a köteget. Külső terhelés hatására a szálak rideg módon eltörnek, azaz egy előre meghatározott törési küszöbig lineárisan elasztikus viselkedést mutatnak. 
Habár a rendszerben a szálak Young-modulusza azonos, azonban $\sigma_{th}$ törési küszöbeiket egy $p(\sigma_{th})$ sűrűség- és egy $P(\sigma_{th})$ eloszlásfüggvény által jellemzett valószínűségi változónak tekintjük, ami rendezetlen tulajdonságokkal ruházza fel az anyagot. 
Az anyagot a hosszanti tengelyével párhuzamosan terheljük - tekintettel arra, hogy a szálak csak az irányukkal párhuzamos terhelést képesek tartani) - aminek hatására az anyag $f$ mértékben megnyúlik, ez a megnyúlás természetesen a külső terhelés függvénye, amit a következő egyszerűsített $\sigma=E_{f}f$ alakban írható fel. A klasszikus modell szerint, amikor az $i$-edik elemre eső $\sigma_{i}$ terhelés eléri az adott elem $\sigma_{th}^{i}$ terhelési küszöbét, akkor eltörik, amit úgy is tekinthetünk, mintha a merevsége nullára esne le.

A modell ezen kiterjesztésében az az újdonság, hogy a szálak törése nem azonnali, hanem a szálak egy fokozatos gyengülésen mennek keresztül. Ezt úgy kell elképzelni, hogy amikor az $i$. szál terhelése eléri annak $\sigma^{i}_{th}$ teherbírását, akkor csupán részleges törést szenved, amit úgy kell elképzelni, hogy az $E_f$ rugalmassági együtthatója a korábbi érték $0\leq\alpha\leq 1$ része lesz. Amikor a terhelés tovább növekszik, akkor a szálköteg a korábbinak megfelelő lineáris viselkedést mutat, csak most már az új $\alpha E_f$ Young modulusszal. A szál így a terhelés növelésével újabb töréseken mehet keresztül. Itt kell megemlítenünk a kibővített modell második paraméterét, amit $k_{max}$-al jelölünk. Ez azt adja meg, hogy a szálak összesen hányszor törhetnek el. A jelöléseket felhasználva $\alpha^{k}E_f$ annak a szálnak a maradék 'merevsége', ami már $k$ alkalommal eltört, továbbá $\alpha^{k_{max}}E_f$ pedig a legkisebb elérhető rugalmassági együttható. jól látható, hogy a paraméterek megfelelő megválasztásával, néhány már ismert határesetet írhatunk le. Amenyiben $k_{max}=1$ és $\alpha=0$, akkor visszakapjuk a klasszikus FBM eredményeit. Amennyiben $k_{max}=\inf$, akkor a tökéletesen (?????) plasztikus esetet kapjuk vissza. 

Egy további fontos kérdés a modellel kapcsolatban, hogy amennyiben egy szál eltörik, akkor a korábbi törési küszöbét tartsa-e meg, vagy mindig új értéket határozzunk meg neki 

\subsection{Magyarázó megjegyzések}
A konstitutív függvény felírásakor azért jön ki az az eredmény ami, mert:
\begin{itemize}
\item Az aktuális teherbírás úgy jön ki, hogy az adott elemek számát csökkentem az eltöröttekkel, majd mivel a törött elemek is megtartják a korábbi terhelésük egy töredékét ezért 
\end{itemize}


