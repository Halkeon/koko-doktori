\chapter{A dolgozat struktúrája}

A következő formában fog kinézni a dolgozat:

Bevezetés:
Dallamos 1-2 oldal a Halász Zoli dolgozatára és a begépelendő szövegekre alapozva,esetleg még a TGábor dolgozatát is figyelembe vehetem.í


Szakterület bemutatása:
- Monte carlo szimulációk
- Sejtautomaták
- Eseményvezérelt szimulációk, az események között a rendszer analitikusan kezelhető, ami annyit jelent, hogy simán kiintegráljuk, majd amikor esemény történik, akkor sztochasztikus szimuláció)

- FBM, extensions of FBM kiterjesztés, kibővítés...
- Fázisátalakulások, statiszikus eloszlások a hatványfüggvények forrása, azok melléktermékségének kiszűrése.


(Egy kis módosítás,az első rész lehet kerek egész, bár nekem fogalmam nincs róla, de a későbbi témákra épít...)

Első komponens: (egy tézis pont, egy cikk 2009)
Lavina dinamikás cikk 2009-es általános (GLS)

Második komponens: (egy tézis pont, egy cikk)
Creep az eseményvezéreltség... ezt mindjárt jól megértem, muhahaha

Harmadik komponens:(két tézis pont, két cikk)
Kétkomponensű rendszerek:
- ELS, van benne valami általánosítás is talán
- LLS,


Referenciák:
Minimum 100
A modellnél inkább monográfiákat
-Alex Hansen 2010, REviews of Modern physics
-Hans 90bol
-Chakrabarti
-Hatano 2012
-2006 bevezetohoz Extension of Fiber Bundle Models... ?Feri Modelling Ciritcal....

Magyar és angol összefoglaló,kb 10-10 oldal
- Bev 0.5 oldal
- Szakt: 1 oldal
- Modsz: 0.5 oldal
- Eredmények: 6-7 oldal
Tézisfüzet: 5 oldal


\section{2017 November a munka ujrakezdese}

Mire van szükség ahhoz, hogy ez a munka tényleg elékszüljön???

1. Definiálni kell azt, hogy milyen legyen a struktúra.
A Zoli dologzatának a struktúrája:
\begin{itemize}
  \item Bevezetés: 1 oldal
  \item Motiváció: 5 oldal
  \item Törés és statisztikus fizika: 3 oldal
  \item Törés modellezési lehetőségei: 6 oldal
  \item A tényleges kutatással kapcsolatos első rész: 23 oldal
  \item A tényleges kutatással kapcsolatos második rész: 34 oldal
\end{itemize}

\subsection{Ötletek}
Arra az ívre kellene felhúzni a dolgozatot, amit feri megpendít a ToresEsFragmentaciosFizika.pdf-es OTKA jelentésben.
A gyakorlatban a szálakkal erősített kompozit anyagok széleskörben használtak a mérnök tudományokban
az építészettől a közlekedésen át egészen a tudomány határmezsgyéjének (frontier?) számító űrkutatásig, űrtechnológiáig.
Ezeken a területeken használt anyagok esetén a minél alacsonyabb tömeg mellett elérhető minél
magasabb teherbírás, megbízhatóság, alkalmazástól függő megfelelő aránya a legfontosabb.
Doktori munkám során nem próbáltam a még oly kecsegtető és ambíciózus célt elérni, hogy eredményeim
a mérnökök izlésének is megfeleljenek. Ugyan speciális esetben az általam továbbfejlesztett
modell kiegészítései a gyakorlat számára közvetlen alkalmazható eredménnyel szolgáltak, a
munkám során a kompozit anyagok törésének jelenség szintű vizsgálatát tűztem ki célul.
A munka első részében azt vizsgáltam meg, hogy milyen hatással van különböző "kölcsönhatás
távolsággal" jellemezhető anyagok esetén az szálakkal erősítés az anyag mechanikai tulajdonságaira, (findings) eredményeimet az x és y fejezet foglalja össze. Érdekes tapasztalatokkal lettünk gazdagabbak a vizsgálatok eredményeként, arról kaptunk képet, hogy milyen formában befolyásolja az erősítés az anyag teherbíró képességét. Miután van egy erős és az elvárt terhelést
többszörösen és tökéletesn teljesítő anyagunk, a kérdés az, mivel a legtöbb esetben ez történik
a gyakorlati használat során: mennyi az anyag élettartama, a műszaki tudományokban régóta ismert
a fáradás jelensége, ahol az anyagot folyamatos, esetleg periódikus szubkritikus terhelés éri,
ami azt jelenti, hogy amennyiben az anyag örökké a kezdeti tulajdonságaival rendelkezne, akkor
soha nem történne semmilyen törés az anyagban, azaz például az autónk örökké ugyanúgy üzemelne.
Látjuk, hogy a gyakorlatban nem ez történik, ugyanis az akár csak rövid ideig tartó periódikus
terhelésnek hosszútávon hatása lehet az anyag minőségre, ezzel idővel az anyag a névleges
teherbírásánál alacsonyabb terhelés hatására is "kritikus" esemény áldozatává vállhat, azaz
egyszerűen csak eltörik.

A mikroszálas kompozitos részt azért nézzem már meg a könyvben... ott akkor is ott volt a 6\% hiába
mondja feri hogy az nem az...

Pilóta és teleshoppos faszom...







