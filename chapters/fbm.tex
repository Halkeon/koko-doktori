\chapter{Szálköteg modell}
\label{chapt:fbm}

\begin{comment}
A szálköteg modelles munkáról a következőket fogom leírni: Ábrákról még lövésem sincs....
1. Motiváció
2. Modell
	- A modell kiterjesztése
	- nevezetes mennyiségek... 
	- a kiterjesztett modell
3. Globális terhelés eloszlás
	Egyik cikk alapján
	- vizsgált mennyiségek, következtetés, eredmények
	- az új univerzalitási osztály, speciális viselkedés észrevétele eloszlástól föggően
	- lavinaeloszlás új eredménye, ennek analitikus meghatározása
4. Lokális terhelés eloszlás
	Másik cikk alapján
	- kritikus viselkedés nagyon alacsony esetben
	- példa a mikroszálas kompozitok esetében
	- lavina eloszlások, skála eredmények vizsgálata 
	- 
\end{comment}


\section{Motiváció}

\section{Modell}
\rhead{\itshape Szálkötegmodellek}
\subsection{Száraz szálkötegmodell}
\subsection{Szálkötegmodell kiterjesztése törhetetlen szálakkal}




\par A rendezetlen anyagok törésének vizsgálatában régóta alkamazott eszköz a
szálköteg modell. A modellt konstrukcióját elõször Peires \verb=\cite{peires1}=
ismertette, amikor 1927-es munkájában gyapjú szálak teherbírását vizsgálta a
segítségével. Az elsõ valószínûségszámítsilag korrekt leírást Daniels-nek
\verb=\cite{daniels1}= köszönhetjük, aki a modell segítségével szálak kötegeinek
átfogó vizsgálatát végezte el feltételezve, hogy az eltörõ szálak a
terhelésüket minden éppen maradt szálra egyenletesen osztják tovább. Az
irodalomban ezt ELS (Equal Load Sharing) néves ismerjük. Az elmúlt évtizedekben
a különféle céloknak megfelelõen több kiegészítése is született a modellnek.
Egy igen fontos kiegészítése a Colman által javasolt idõfüggés volt, ahol a
szálak teherbírása az idõnek monoton csökkenõ függvénye. Ezzel a megközelítéssel
fáradás és creep okozta törések is vizsgálhatók lettek.
\par A szálköteg modellek két fontos a modellalkotással szemben támasztott
kihívásnak is megfelenek. Egyrészt ha anyagok teherbírását, sérülékenységét
szeretnénk vizsgálni, akkor olyan modellt kell alkotnunk ami elegendõen
ralisztikus ahhoz, hogy segítségével kellõ részletességgel írhassuk le
az anyag mikroszerkezetét, és lokális feszültségviszonyait. Ha sikerült ilyet
alkotnunk akkor képesek lehetünk arra, hogy meghatározzuk a kapcsolatot
az anyag mikroszkópikus paraméterei és makroszkópikusan mérhetõ
jellemzõi között. Természetesen a realisztikusság mellet fontos, hogy a
modell bizonyos feltételek mellett analitikusan is kezelhetõ legyen, és így a 
segítségével kapott eredmények összevethetõk legyenek a
statisztikus fizika eddigi fejezeteivel vagy beilleszthetõk legyenek az
eddigi elméletekbe, különös tekintettel a kritikus
jelenségek és a fázisátalakulások elméletére. 
\par A szálköteg modellek mindkét 
szempontból kitûnõ választásnak bizonyultak, mivel egyszerûségük ellenére
jól megragadják a folyamatok legfontosabb jellemzõit és bizonyos határesetekben
analitikus számításokra is haszálhatók.

\section{A szálköteg modell megalkotása}
\par Mint minden modell a szálköteg modell is különbözõ egyszerûsítõ
feltételeket fogalmaz meg annak érdekében, hogy kezelhetõvé tegye a modellezni kívánt
problémát. Ebben az esetben a következõ egyszerûsítõ feltevéseket tesszük:
\subsection{Diszkretizáció}
\par Ebben a modellben a rendezetlen anyagot szálak egy halmazának feleltetjük
meg. Legyen a diszkretizációhoz használt szálak száma $N$, amit valamilyen
reguláris rácson helyezünk el, mint pl.: háromszög-, vagy négyzetrácson. 
Mivel a szálak ismert tulajdonsága, hogy csak húzófeszültség ébredhet bennük,
ezért egy szálakból épített modellre általában csak a szálakkal párhuzamos
terhelést helyezhetünk, de legalábbis csak a szálak irányába esõ komponens
hatását vizsgálhatjuk.
\subsection{Szálak törési törvénye}
\par Mivel anyagok törését, vagy sérülését szeretnénk vizsgálni a modellünkel,
ezért a diszkretizáció során meghatározott szálakat valamilyen módon el
szeretnénk majd törni. Az egyszerû szálköteg modellben - mostantól DFBM (Dry
Fiber Bundle Model)- a szálak a növekvõ külsõ terhelésre ridegen reagálnak, ami
annyit jelent, hogy amennyiben a szálban ébredõ feszültség a deformációnak
lineáris függvénye, akkor egy elõre meghatározott $\sigma_i^{th}$,
$i=1,\ldots,N$ törési küszöb elérése után a szál eltörik és a rá esõ feszültség
nullára esik. Ezt a kapcsolatot általában a
\begin{equation}
\label{equ:hook1}\varepsilon=\frac{\Delta l}{l},\ \varepsilon=\frac{1}{E}\sigma
\end{equation} 
lineáris Hook törvényel szokták leírni, ahol $\varepsilon$ a szál deformációja,
$\sigma$ pedig a benne ébredõ feszültség. Az $E$ egy anyagi jellemzõktõl függõ
konstans, az anyag Young modulusza. Ez a mennyiség határozza meg az anyagunk
rugalmas tulajdonságait. Az $E$ értékét egységesen minden szálra azonosnak
határozzuk meg. Az egyszerû DFBM-ben a szálak soha nem regenerálódnak, így ha
valamelyik szál egyszer eltört, akkor törött is marad.
\par A különféle kerámiák törési törvénye meglehetõsen jól egyezést mutat a 
fenti \ref{equ:hook1} egyenlettel definiált Hook törvénnyel. A szálköteg
modelleke ebbõl adódóan jól használhatók kerámiák sérülésének, törésének a
vizsgálatára.
\subsection{Terhelés elosztás}
\par Miután a modell alkalmazása során a szálakat valamilyen módon el fogjuk
törni, az általuk tartott terhelést valamilyen módon tovább kell osztani a még éppen
maradt szálakra. Azt, hogy az éppen eltört szál terhelését milyen
távolságon belül, és milyen módon osztom tovább, azt a terhelés osztás módja
határozza meg. Az irodalomban fellelhetõ, ezzel a modellel elért eredmények
jelentõs többségét két szélsõséges esetre határozták meg. Az egyik véglet a
végtelen hatótávolságú kölcsönhatás, ami lényegében azt jelenti, hogy minden
még épp szál azonos módon részesül a törött szál által tartott terhelésbõl. Ezt
a szélsõséget nevezik az irodalomban globális (GLS - Global Load Sharing), 
illetve egyenlõ terhelés elosztásnak (ELS - Equal Load Sharing), habár ez
bizonyos esetekben nem azonos a GLS-el. A másik véglet esetén a kölcsönhatás
hatótávolsága annyira rövid, hogy csak a közvetlen még épp szomszédok kapnak a
szál által tartott terhelésbõl. A közvetlen szomszéd jelentése rácsfüggõ,
négyzetrács esetén a klasszikus Neumann-szomszédságot értjük alatta.
\par A globális eset a szálköteg modell átlagtérközelítésének felel meg,  ami a
jellemzõk analitikus meghatározására ad lehetõséget. A globális kölcsönhatás
tulajdonképpen irrelevánsá teszi a topológiát 
\par Mivel LLS esetén a lokalitás miatt kilakulhatnak nemtriviális térbeli
korrelációk, ezért ennek a végletnek az analitikus vizsgálata csak ritkán 
valósítható meg. A legtöbb LLS-el kapcsolatos eredmény nagyszámú számítógépes
szimuláción alapszik.
\par A fentiek tükrében egy bonyolultabb probléma esetében
is érdemes elõször a GLS megközelítéssel kezdeni a vizsgálatot, mert a szálköteg legfontosabb
jellemzõ mennyiségeit zárt analitikus alakban határozhatjuk meg vele.
\subsection{Rendezetlenség a modellben}
\par A szálköteg modelleket rendezetlen anyagok vizsgálatára használjuk,
felmerül tehát a kérdés, hogy a rendezetlenség milyen módon jelenik meg a modellben. A
topológiát a kezelhetõség miatt általában négyzet, vagy hasonlóan szabályos
szerkezetû rácson valósítjuk meg, és mivel a szálak rugalmas jellemzõjét az $E$
Young-moduluszt egységesen egyformának határoztuk meg, nem marad más
lehetõségünk, minthogy a szálak törési küszöbei legyenek a rendezetlenség
hordozói. 
\par Az egyes szálak eltöréséhez szükséges $\sigma_i^{th}$ lokális terhelések
értékei egy elõre definiált $p(\sigma^{th})$ valószínûségi sûrûség  és
$P(\sigma^{th})=\int_{\sigma^{th}_min}^{\sigma^{th}_max} p(x) dx$ kumulatív
eloszlás függvényt követnek. A törési küszöbök rendezetlenségének mértéke
határozza meg a modellezni kívánt anyag, rendszer rendezetlenségének a mértékét.
\par A legtöbb esetben a vizsgálatot érdemes az analitikusan kezelhetõ $0$ és
$1$ közötti egyenletes eloszlással kezdeni, aminek a sûrûség és eloszlás
függvénye
\begin{equation}
\label{equ:uniform1}
p(\sigma^{th})=1,\ \ \ P(\sigma^{th})=\sigma^{th}
\end{equation}
alakban adódik. Egy másik széles körben alkalmazott eloszlás az FBM modellekben
a kétparaméteres Weibull eloszlás, amit a
\begin{equation}
\label{equ:weibull1}
P(\sigma^{th})=1-exp\left[-\left(\frac{\sigma^{th}}{\lambda}\right)^m\right]
\end{equation}
eloszlásfüggvény jellemez. Az $m$-et Weibull indexnek, a $\lambda$-t skála
paraméternek nevezzük. Ennek az eloszlásnak a vizsgált probléma szempontjából
két fontos elõnye is van. Egyrészt a tapasztalatok szerint
megfelelõ paraméterek mellett viszonylag reálisan írja le a vizsgált
anyagok terheléseloszlás viszonyait, másrészt egyetlen paraméter, az $m$
segítségével állítható az aktuális módszerben a rendezetlenség mértéke. A
gyakorlatban elõforduló anyagok esetén az $m$ paraméter értéke $2$ és $10$ közé
szokott esni, ahol a kisebb szám a nagyobb rendezetlenséget jelöli. 




Univerzalitási osztályok szignifikánsan különbözo szálak keverésével eloállított szállkötegekben

Anyagok külso terhelés hatására történo törése vagy sérülése a tudomány egy nagyon érdekes és izgalmas
területe, amelynek eredményeit széleskörben alkalmazzák a megvalósuló technológiák esetében.

Az elmúlt két évtized során a statisztikus fizika eszköztárának alkalmazásával a mérnökök és fizikusok
betekintést nyerhettek az anyagok törése mögött rejlo folyamatokba. A kutatások eredményeként kiderült,
hogy a törési folyamatokat nagyban befolyásolja az anyagok belso szerkezetének heterogenitása illetve
ezen heterogeneitások hasonlósága, illetve különbözosége.
A közelmúltban több különbözo sztochasztikus törési modellt is alkottak az anyagok szerkezetében tapasztalható
rendezetlenség hatásainak a megismerésére. Ilyen a szállkötegmodell is - (Fiber Bundle Model) a továbbiakban FBM -
illetve különféle biztosítékok, rugók rácsmodelljei.
Ezeket a modelleket felhasználva a tudományos közösség számtalan numerikus szimuláció és analitikus
vizsgálat segítéségével kapcsolatot talált a rendezetlen anyagok makroszkópikus törése és a fázisátalakulások
és kritikus jelenségek elmélete között, amelyek számos univerzális tulajdonsággal rendelkeznek a vizsgált
anyag konkrét anyagi jellmzoitol függetlenül. Az egyik ilyen fontos felismerés az volt, hogy lassan növekvo
külso terhelés hatására bekövetkezo makroszkópikus törést úgynevezett hirtelen löketekben jelentkezo jelenségek
elore jelezhetik, amelyeket a lokális törések lépcsozetes természetére lehet visszavezetni. 

Példám
Ezt a viselkedést
jól szemlélteti a következo gondolatkísérlet. Helyezzünk el kölönbözo vastagságú nejlonzacskókat egymás
alá, ez a különbözo vastagság szemlélteti az anyagok mikroszerkezetének rendezetlenségét. Ezután helyezünk egy adott
tömegu testet a legfelso nejlonra. Tegyük fel hogy ez a test olyan, hogy távirányítással növelhetjük a tömegét.
Ennek a gyakorlati megvalósítását a mérnökökre bízom. Tehát addig növeljük finoman a tömegét ennek a testnek amíg át nem 
szakítja a felso nejlont. Amennyiben az alatte lévo nejlon erosebb mint amit korábban átszakítottunk, akkor ez megfogja
a testet és így azt újra növelnünk kell, azonban az is elofordulhat hogy a következp nejlon gyengébb így a test átzuhan rajta
majd a következon, amíg el nem érünk egy erosebbhez. Ahogy növeljük a tömeget a folyamatosan átszakított nejlonok száma
átlagosan egyre több lesz. Ha valamilyen módszerrel ezeknek a folyamatos zuhanásoknak az idejét mérni tudnánk
(a valóságban a mérés tárgya nem feltétlenül ido dimenziójú) akkor ismerve az elméletbol hogy a kritikus jelenség elott
a mér mennyiségünk maximummal rendelkezik, akkor elojelezhetjük ezt. cite{3-4}

Mivel általában ezeket a löketeket kísérletileg is mérni lehet például az akusztikus emisszión alapuló technikákkal (mi az az 
akusztikus emisszió?) ezek a jelnség elojelzo markerek lehetoséget adnak a jelenség elojelzésére. A "löketek" eloszlása
analtikusan bizonyított, hogy hatvány függvény alakú - amit a szimulációk és kísérletek is alátámasztottak - méghozzás
a jelenségek széles körét tekintve állandó exponenssel. cite Az azonos exponenssel rendelkezo jelenségek egy úgynevezett 
univerzalitási osztályt alkotnak. 

A közelmúltban az univerzalitási olytályok robosztusságát több munka is vizsgálta olyan esetekben amikor az anyagok rendezetlenségét
jellemzo különbözo eloszlásokat keverték egymással, illetve egy másik esetben az eloszlás azonos volt, de az egyik eloszlást eltolták 
a másikhoz képest. Változást a löketek eloszlásában azonban csak akkor sikerült elérni amennyiben egy véges alsó határt szabtak meg a 
az eloszlásban lásd cite. Növelve az alsó küszöböt az eloszlás exponense megváltozik. Egy átmenetnek lehetünk tanui ami az 5/2-es 
exponens átmegy egy alacsonyabb 3/2-be.

Divakaran és Dutta megvizsgálták egy olyan FBM kritikus viselkedését amelyikben a szálak terhelés határai két egymáshoz képest eltolt
egyenletes eloszlásból származtak. Az általunk vizsgált rendszert úgy is interpretálhatjuk mint a Divakaran féle modell kiterjesztését
egy végtelen távolságú eloszlásokra gapu/résu. Azt egyébként elégséges megszabnunk feltételnek, hogy az erosebb szálak közül a leggyengébb
szállnak is erosebbnek kell lennie, mint az összes gyenge szállnak együttvéve.

Ebben a munkában megvizsgáljuk az anyagban megjeleno eros heterogenitás hatását a törés folyamatára. Feltételezzük, hogy a rendszer 
két komponensbol épül fel. Az egyik komponenst egy ero eloszlással jellemezhetjük, míg a másik komponenst törhetetlennek tekintjük.
Változtatva a két komponens $\alpha$ arányát globális terhelés eloszlás mellett (Global Load Sharing továbbiakban GLS), megmutatjuk analitikus
eszközökkel, hogy a törhetetlen elemeknek jelentos hatása van a rendszer viselkedésére mind a mikor és makro méretek területén.
Érdekes módon találtunk egy kritkus $\alpha_c$ keverési arányt amely mellet két különbözo tartomány között átmenet valósul meg. Azon
 $\alpha$ értékek esetén amelyek az említett $\alpha_c$-nél kisebbek, a makroszkópikus viselkedést leíró konstitutív görbe egy
 maximummal rendelkezik. a burst méret eloszlás hatványfüggvény eloszlású méghozzá az irodalomból ismert $\tau=5/2$-es  kitevovel.
Amennyiben az arány meghaladja $\alpha_c$-t a makroszkópikus válasz monoton növekvové válik és a burst méret eloszlás pedig
habár most is hatványfüggvény lesz de az exponens az átlagtér megoldás $5/2$ értékérol $\tau=9/4$-re vált. A makroszkópikus válasz
analitikus vizsgálatával megmutatjuk, hogy az átmenet olyan eloszlások esetén valósul meg, amelyek egy pontosan egy maximummal
 és egy inflexiós ponttal rendelkeznek, s így egy új univerzalitási osztályt alkotnak a breakdown jelenségben???...
 
 Modell:
 Az FBM modell esetén a modell rendszerünk N szállból áll amelyeket a szálak irányával párhuzamosan terhelünk. Egy $\sigma_0$ külso
 terhelés hatására a szállak lineárisan elasztikus választ adnak. Minden szállat azonos E=1 Young modulusszal jellemzünk.
 A jelentos heterogenitás modellezéséhez feltételezzük, hogy a szállköteget két erosségében jelentosen különbözo
 száll alkotja, A szállak egy $0\leq\alpha\leq 1$ hányadát "erosnek" tekintjük, ami annyit jelent, hogy ezek a szállak törhetetlenek.
 A maradék $1-\alpha$ hányada "gyenge" száll, ami annyit tesz, hogy eltörnek amennyiben az egyes szálakon fellépo $\sigma$ feszültség
 meghaladja az adott száll $\sigma_{th}^i, i=1..N_w$ teherbíróképességét, ahol $N_w=N*(1-\alpha)$. A gyenge szállak teherbíróképessége
 egy $p(\sigma_{th})$ suruség és $P(\sigma_{th})$ eloszlásfüggvény által jellemzett eloszlást követ. 
 Miután egy gyenge száll eltörik, az addig rajta lévo terhelést szét kell osztani a még épp szállakra. Az egyszeruség és analitikai
 kezelhetoség miatt egyenlo terhelés eloszlást feltételeztünk ami egy végtelen hatótávolságú kölcsönhatásnak felel meg, s ez annyit jelent,
 hogy minden száll a teljes terhelési szimuláció alatt azonos terhet visel és így nem tapasztalunk növekvo feszültségkoncentrációt a
 sérült részek környezetében.
 
 Ilyen feltételek mellett a rendszer makroszkópikus viselkedését leíró konstitutív egyenlet a következo alakot veszi fel:
 
\beq
\label{equ:consti}
\sigma_0=(1-\alpha)[1-P(\sigma)]\sigma+\alpha\sigma
\eeq
 
 Ezt az egyenletet más alakba is írhatjuk, ami ..... A $\sigma_0$ érték az átlagos külso terhelés a rendszeren, míg a $\sigma$
 az egyes szálakon fellépo feszültség ami a szálak aktuális $\epsilon$  fajlagos megnyúlásával a következo kapcsolatban áll
 $\sigma=E\epsilon$
 
 Az xy egyenlet elso kifejezése a még épp gyenge szállak által viselt terhelésért felelos, az egyenlet második tagja pedig 
 a törthetetlen szállakon lévo terhelést jelenti. A következo számításokat két különbözo terhelés eloszlásra is kiszámítottuk.
 A két vizsgált eloszlás az egyenletes és a Weibull a megfelelo $P(\sigma)=\sigma$ és $P(\sigma)=1-exp[-(\sigma/\Lambda)^m]$.
 
 Az xy ábrán látható a rendszer konstitutív viselkedése a két vizsgált eloszlás esetében. Az $\alpha=0$ esetben visszakapjuk a szállkötegmodell
 irodalomból jól ismert megoldásait. Az FBM modell klasszikus megoldásai általában pontosan egy parabólikus maximummal rendelkeznek mind
 az egyenletes mind a weibull eloszlás esetén. A görbék maximum helye egy kritikus $\epsilon_c=1/E\sigma_c$ deformációt és 
 egy kritikus $\sigma_0^{c}(\sigma_c)$ terheléshatárt definiál. A rendszer katasztrófikus eseményen megy keresztül amikor eléri
 ezeket a kritikus értékeket.
 Nem zérus $\alpha$ értékek esetén kelloen nagy $\sigma$ terhelés esetén minden gyenge száll eltörik és így az akármilyen egyenlet elso tagja 
 nulla lesz és így a makroszkópikus választ tisztán a törhetetlen szállak által tartott terhelés képviseli, amit az xy ábrán
 abból is láthatunk, hogy minden konstitutív görbe nem nulla $\alpha$ esetén egy $E\cdot\alpha$ meredekségu egyeneshez tart.
 Ha alaposabban megvizsgáljuk az ábrát észrevehetjük, hogy alacsonyabb $\alpha$ értékek esetén a $\sigma_0{\sigma}$ görbe lokális maximuma
 továbbra is megfigyelheto, azonban növekvo $\alpha$ értékekkel a $\sigma_c(\alpha)$ helye és $\sigma_0^c{\alpha}$ értéke monton növekszik.
 Ha tovább vizsgáljuk a görbék viselkedését észrevehetjük, hogy egy jól definiálható $\alpha_c$ értéknél nagyobb $\alpha$ értékek
 esetén eltünk a lokális maximum és a konstitutív görbe monton növekvové válik, azaz $\sigma_0/d\sigma>0$. A maximum helyét egyszeruen meghatározhatjuk
 a $\frac{d\sigma_0}{s\sigma}|_{{\sigma_c}=0}$ egyenlet eredményeként. Ennek az eredménye a következonek adódik:

\beq
\label{equ:consti2}
\frac{1}{1-\alpha}=P(\sigma_c)+\sigma_c p(\sigma_c).
\eeq

 
 Ha a fenti egyenletet $\sigma_c$-re mint $\alpha$ függvényére megoldva a $\sigma_0^c$ a maximum hely értéke egyszeruen adódik ha a meghatározott
 $\sigma_c(\alpha)$ értéket az xy(alap const egy) egyenletbe helyettesítjük be.
 
 Mivel a $\sigma_0(\sigma)$ konstitutív egyenlet deriváltjának a $\sigma_{in}$ inflexiós pontban minimuma van ezért az xy(1/1-alpha) egyenletnek maximuma
 lesz $\sigma_c=\sigma_{in}$ inflexiós pontban. Ebbol következik hogy a második egyenlet csak addig oldható meg $\sigma_c(\alpha)$ értékre
 amíg $\alpha\leq\alpha_c$, ahol $\alpha_c$-t a $\sigma_c$ segítségével úgy definiáljuk mint azt az $\alpha$ értéket ami mellett a konstitutiv görbe
 inflexiós pontja és maximumhelye egybeesik, azaz $\sigma_c(\alpha_c)=\sigma_{in}$. Ahhoz, hogy ezt használhassuk be kell látnunk, hogy az inflexiós pont helye
 független $\alpha$ értékétol. Ez azonban könnyen belátható ha megtekintjük a constitutiv görbe második deriváltját ami a következőnek adódik:
 
\begin{equation}
\label{equ:consti_derivalt}
 \frac{\partial^2\sigma_0}{\partial\sigma^2}|_{\sigma_c}=-(1-\alpha)[2p(\sigma)+\sigma p'(\sigma)].
\end{equation}
 
 A Weibull eloszlás esetén a $\sigma_c(\alpha)$ függvényt nem lehet zárt alakban meghatározni, azonban két $\alpha$ érték $\alpha=0$ és $\alpha=\alpha_c$ esetén egyértelmuen
 meghatározhatjuk $\sigma_c$ értékét. A számítások eredménye $\alpha=0$ esetén a következonek adódik
 
\beq
\sigma_c(\alpha=0)=\Lambda(1/m)^(1/m),
\eeq
 
 míg $\alpha=\alpha_c$ esetén amikor a maximum hely egybe esik az infelxiós hellyel a következő eredményt kapjuk:
 
\beq
\label{equ:inflex}
	\sigma_{in}=\Lambda[(1+m)/m]^{1/m},
\eeq

 MEGJEGYZÉS: szerintem ezeket a képleteket bővebben is kifejthetném... sztem
 
 Ezután az $\alpha_c$ értéket úgy kaphatjuk, hogyha a fenti eredményeket behelyettesítjük a konstitutív görbe eredeti xy egyenletébe, ami a következőnek adódik
 
\beq
\label{equ:consti3} 
\alpha_c=m\exp{-(1+m)/m}/[1+m\exp{-(1+m)/m}]
\eeq


\section*{Jegyzetek az fbm hez}
Az LLS esetben az átlagos klaszterméretnek maximuma van és ennek van maximum helye, maximum értéke ... ezeket az értékeket egész pontosan meghatározhatjuk A Stauffer Scaling theory of percolation-ben a leírtak alapján 11.oldal numerical methods....
