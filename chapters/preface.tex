\chapter*{Előszó}
\label{chapt:preface}

Ez a dolgozat a Debreceni Egyetemen megkezdett doktori munkám eredményeit foglalja össze. A munkát Dr. Kun Ferenc irányítása alatt végeztem. Az ő előadásain találkozhattam egyetemi tanulmányaim alatt  a számítógépes fizikával és a különböző fizikai rendszerek szimulációs vizsgálatának a lehetőségével. 

Másik témavezetőm Dr. Vertse Tamás előadásain a numerikus módszerekkel kerülhettem közelebbi kapcsolatba, amely terület alapjául szolgál minden szimulációs programnak, és minden olyan területnek amely a szimulációk eredményére épít.

Az informatika és a számítástechnika meglátásom szerint ma az a tudomány terület amely a matematika mellé felzárkózva, a matematika egy évszázaddal ezelőtti szerepét igyekszik betölteni, abban a tekintetben, hogy olyan eszközöket szolgáltat a többi tudomány számára, amellyel addig le nem írható, meg nem vizsgálható területekre kalauzolja a kutatókat. 

A ókortól kezdve egészen a XX. század kezdetéig az egyes tudományterületeket jól definiált határok választották el egymástól, azonban a tudomány művelői maguk több tudományterülettel is foglalkoztak egyszerre. \cite{introd_simonyi1998} Ahogy a felhalmozott tudás nőtt, úgy szűkült azon területek száma amivel egy ember képes volt foglalkozni. Ez mára oda jutott, hogy már csak egy tudományterület egészen kis szeletével foglalkozik a legtöbb kutató, azonban időközben a jól definiált határok eltűntek, ezért megint az a helyzet állt elő, hogy több területtel foglalkozunk egyszerre.

A doktori munkám során én az informatika, számítástudomány, számítógépes fizika, statisztikus fizika és a klasszikus fizika eredményeire alapozva végeztem a kutatásaimat, az eredményeim alapvetően a számítógépes, illetve statisztikus fizika tárgykörébe tartoznak. Mindamellett, hogy az eredmények megszületéséhez jelentős mennyiségű kód megírása, szimulációs technikák és - a hatékonyság növelés érdekében - új technológiák megismerése volt szükséges.

Szeretnék köszönetet mondani témavezetőimnek, kollégáimnak, a családomnak és barátaimnak, a segítségükért és türelmükért amellyel ennek a dolgozatnak a megszületéséhez hozzájárultak. 
