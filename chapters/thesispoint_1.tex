\chapter{Talalo leirasa a lavina dinamikanak}
\label{chapter:thesispoint_1}
A fejezetben tárgyalt eredmények a Physical Review E 80, 051108-ban publikált cikk
kibővített fordítása.

\section{Bevezetés}
A rendezetlen (nem is, hanem heterogén) anyagok terhelés hatására bekövetkező károsodásának és törésének vizsgálata
egy régóta kutatott izgalmas tudományterület, amelynek eredményei széles körben alkalmazhatóak
a gyakorlatban.\cite{thsp1_herrmann1990,thsp1_chakrabarti1997} Jól ismert jelenség az, ahogy
az ilyen heterogén anyagok egymást követő törési hullámokban (bursts) károsodnak lassan és folyamatosan növekedő terhelés hatására. Ez azt jelenti, hogy nyugodt, törés mentes és
lokálisan korrelált törésekkel teli szakaszok váltogatják egymást.  (itt 4-15 referencia)
Mivel ezek a hullámokban, lavinákban érkező törések rugalmas hullámokat indukálnak az anyagban,
amelyeket kellően érzékeny eszközökkel zajként mérhetünk. (Akusztikus emisszió, wiki, műműszaki tudomány) 



%egy gyakorlat számára i izgalmas tudományterület.
%impact
%érdeks??? magával ragadó rettentő izgalmas
%egy széles körben alkalmazottak a
%A rendezetlen anyagok törése és kársodása egy széles gyakorlati hasonnal kecsegtető
%igen érdekes tudományterület.
\section{•}