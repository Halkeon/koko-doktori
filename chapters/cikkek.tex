\chapter{Cikkek}
\section{Peirce, F. T. - 1926 J. Textile}
Link: 	http://dx.doi.org/10.1080/19447027.1926.10599949

Pierce egy teljes cikksorozatot írt a gyapott szálakból készített kötegek mechanikai, kísérleti vizsgálatáról, az azokból elvonható következtetésekről.
1926-os cikksorozatában foglalkozott a következőkkel:

\subsection{1. cikk}
Felismerte, hogy a mérései eredményét erősen befolyásolja az éppen tesztelt fonal minősége, illetve a gyártási mód ahogyan azt előállították, a fonál kora, amely külső tényezőket mind figyelembe kell venni mielőtt az ember messzemenő általánosításokat próbálna levonni a kísérleti eredményekből.

Munkájának egyik célja egy olyan vizsgálati eljárás meghatározása amellyel az akkoriban megjelenő új fajta mesterséges selymek, vagy új nemesített pamut szálak minősége kvalitatíve összevethető. 
Egy kísérlet a következő jellemzőket vizsgálhatja, a szál minősége, a nyers gyapot minősége, vagy a szál csavarási technika hatásossága.
Ahhoz hogy például két gyapotszál minőségét összehasonlíthassuk, mindkettőt ugyanazon gépen, ugyanazon gyártási folyamatok és mechanikai körülmények mellett kell előállítani, máskülönben az összehasonlítás értelmetlen.

A cikksorozatban a következő külső, illetve gyártási technikából eredő körülményekre kell figyelmet fordítani
\begin{enumerate}
\item Mintavételezés módja - hogy a báláról megfogalmazottak mérvadóak legyenek
\item A vizsgálat természete - 
\item Az eszköz hatékonysága
\item Légköri körülmények (relatív páratartalom, hőmérséklet)
\item A vizsgált mintadarab hossza
\item A törés rátája
\item Az eredmények megfelelő kifejezése
\end{enumerate}

Megjegyzendő, hogy a pamut szálak nagy fokú rendezetlensége megnehezíti a vizsgálati módszereket, éppen ezért lépett életbe később ezeknek az eredményeknek a statisztikai értelmezése. Már ő is megfigyelte, hogy minél nagyobb egy gombolyag mérete annál nagyobb a rendezetlensége, lásd a végtelen méretű szálköteg teherbírása 0.




\subsection{2. cikk}
Munkáján node mifene
