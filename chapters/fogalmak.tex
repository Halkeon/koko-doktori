\chapter*{Fogalmak Jelölések}
\label{chapt:fog}

\begin{itemize}
\item HTR: Hosszú távú rendezettség
\item FBM: Fiber Bundle Model (Szálköteg Modell)
\item DFBM: Dry Fiber Bundle Model (száraz szálköteg modell)
\item CDFBM: Continuous Damage Fiber Bundle Model / Fokozatosan károsodó szálkötegmodell
\item Extenzív mennyiség: Ha a rendszerekre nézve additív, (például tömeg, ha A és B alkotta rendszer tömege az A és B rendszerek tömege, rendszer méretével arányos, pl. belső energia, entrópia, stb...
\item Intenzív mennyiség: homogén rendszer esetén a rendszer méretétől függetlenek
\item
\item
\item
\item
\item
\item
\item
\item
\item
\item
\item
\item
\item
\item
\item
\item
\item
\item
\item
\item
\item
\item
\item
\item
\item

\end{itemize}

Illetve Angol-Magyar fordítása egyes jelenségeknek:

\begin{itemize}
\item algebric decay: ez majd kiderül a cikkből remélhetőleg...	hmmm power law???
\item boundary node: peremcsomópont, határcsomópont
\item breakdown: roncsolódás, elromlás (hátha idővel lesz ez jobb is)
\item burst: hullám,
\item cotton: pamut, gyapot
\item crackling-noise: károsodási zaj
\item creep rupture:
\item damage: károsodás, roncsolódás
\item damage process:károsodás(i folyamat)
\item damage enhanced:
\item draw: húz, gondolom szálat
\item dynamics: folyamatok
\item extensibility: nyújthatóság
\item elasticity: rugalmasság, hajlékonyság bár itt nem
\item flex: len, (kóc a rövidebb szálú len, tömítéshez), mind az alapanyag mind a szövet jó víz szívó tulajdonsággal rendelkezik.
\item heterogenous materials: heterogén anyagok?, erre valami jobb magyar fordítás kéne, ez nem a granular matter megfelelője gondolom, bár lehet köztük kapcsolat
\item gradual: lépcsőzetes, fokozatos
\item lea: mező, de a unit for measuring lengths of yarn, usually taken as 80 yards for wool, 120 yards for cotton and silk, and 300 yards for linen, illetve: a measure of yarn expressed as the length per unit weight, usually the number of leas per pound
\item linearly elastic behavior: lineárisan rugalmas válasz, pfff
\item macrospcopic hardening: az anyag makroszkópikus felkeményedése? jó ez így?
\item macroscopic failure: "tönkremegy", használhatatlan lesz, makroszkópikus törés (by feri)
\item plastic: képlékeny
\item quasibrittle: kvázi
\item reinforced: megerősített
\item sesal:szizálkender, egy pálma fajta aminek a szálaiból kötelet gyártanak
\item shear: kúszás
\item skew / skewness: itt a Pierce fele cikkben arra utal, hogy nem szabalyos, nem szimmetrikus az eloszlas, hanem ferde, a fvv nem szimmetrikusságának a mértéke, (Mean-Mode)/SD (standard deviation)
\item sound measure: megbízható mérőszáma valaminek
\item sound: mebízható, szilárd, erős egészséges, ép, józan, tartós, mélyen (határozó szóként)
\item spinning: fonás, egy szál készítési technika, több módozata is ismeretes, turbinás hagyományos, stb... elemi szálakból azok párhuzamosításával
és összesodrásával készítünk hosszabb szálakat, végül így kapjuk a fonalat.
\item stiffness: merevség (ezen meg dolgozni kelll)
\item technical applications: gyakorlati alkalmzás...
\item worsted: gyapjú (woolen)
\item yarn: fonál, legyen az akár gyapjúból akár gyapottból

\item
\item
\end{itemize}


Mérnöki fogalmak: fáradási szilárdság...
szilárdság
