%\chapter{Anyagok törésének vizsgálata molekuláris dinamikai módszerekkel}
\chapter{Fragmentáció lövedék becsapódás hatására}
\label{chapt:md}

\begin{comment}
Nos egyelőre annyit tudok, hogy itt egy picit bővebben ki kellene fejteni az ami a kb 10 oldalas cikkben benne van, meg kifejteni a modellt amit használok. Meglátjuk, hogy sikerül.
1.
Motiváció

2.
Modell
Gábor és Feri cikke alapján leírni a felhasznált modellt.
Ábrák:
-A rendetzetlenséget szemléltető ábra
-Hertz erő  a polygonokkal
-A rudak kölcsönhatását bemutató ábra

3.
Vizsgálatok  a fragmensek sebesség és tömegeloszlása

4.
Következtetések

\end{comment}

\section{Tartalom}

	A fragmentációs rész tartalmazza a következőket:
	1. Fragmentáció bevezető elmélet
	2. Modell
	3. 

\section{Mas}
A fragmentáció mint a szemcsés anyagok törésének módozatait összefoglaló jelenség egy a természetben mindenütt jelenlévő folyamat ami sok a természetben és ipari környezetben előforduló folyamat hátterében áll. Igen széles méret
tartományon figyelhetjük meg ezt a folyamatot, az égitestek mérettartományától - ahol a fragmensek a nagyobb égitestek összeütközése
eredményeként létrejött különböző méretű aszteroidák - a geológiai jelenségek mérettartományán keresztül egészen a 
a nehéz magok bomlási termékeinek a mérettartományáig. CITE
A fragmentációs folyamatok 


ami a szemcsés anyagok törésének különbözo módozatait összefoglaló fogalom
egy mindenütt jelenlévo folyamat ami sok természeti jelenség és ipari folyamat hátterében
áll. Ez a jelenség az aszteroidák mérettartományától(amelyek nagyobb égitestek ütközése 
révén jönnek létre) a geológiai jelenségek mérettartományán keresztül egészen a nehéz magok 
bomlásai termékeinek mérettartományáig elofordul. %(\cite{ACTAPHYS.1}) 
A fragmentáció egyik legmeglepobb tulajdonsága hogy az eloálló fragmensek méreteloszlása
hatványfüggvényt követ függetlenül a mikroszkópikus kölcsönhatások milyenségétol és a releváns
hosszúság skálától.

A doktori munkám során először ezzel a fragmentációs problémával találkoztam. 
A fragmentációs folyamatok a természetben széres méretskálán fordulnak elő. A hagyományos törési folyamatoktól abban különböznek, hogy
fragmentáció esetén a test nem csak pár néhány tíz, hanem több nagyságrenddel több darabra törik szét.  
\section{Motiváció}

\section{A modell}
\section{Fragmensek tömeg és sebesség eloszlása}
\section{}


Az elmúlt évtizedek elmléleti és kísérleti kutatásai a hatványfüggvény fragmens méreteloszlás
okait keresték. Megpróbáltak kapcsolatot találni a fragmentáció és a kritikus jelenségek elmélete illetve
a fázisátalakulások elmélete között. Jóval kevesebbet tudunk a fragmensek sebességeirol
illetve a sebességek és a fragmens tömegének kapcsolatáról, ami fontos szerepet játszik a naprendszerbeni
meteroid felhok kialakulásában és a planetezimális növekedésben (bolygóképzodés). Ezek az ismeretek
segítségünkre lehetnek a Föld körüli pályán keringo az emberi tevékenység (urbe juttatott muholdak
magasre juttatott indító fokozatok, ballisztikus rakéták) eredményeként odakerült urszemét
idofejlodésének elorejelzésében.

A fragmensek sebességének kísérleti méréséhez 2-3 nagysebességu kamera megfeleloen idozített használatára 
van szükség. Ahhoz hogy ezek statisztikailag értékelheto eredményeket adhassanak, ahhoz sok fragmens 
mozgását kell egyszerre követni. Ennek a lehetosége a közelmúlt technikai fejlodésével vált
elérhetové a kutatások számára. T. Kadono és kollégái egy vékony üveg lemez lövedék általi fragmentálódását
tanulmányozták nagysebességu képkészítési eljárás segítségével \verb=\cite{fragment 7-es}=. A kísérlet során egy Pyrex
üveglemezbe lottek egy nagy sebességu lövedéket az üveg síkjával párhuzamosan. A lövedék üveglapba hatolását
egy céltárgy elején elhelyezett lövedék blokkolóval fogták meg, s így a lövedék csak egy nagy energiájú 
nyomáshullámon keresztül fejtett ki hatást az üveglapra. A fragmentációs folyamatot két nagysebességu kamera
segítségével követték nyomon. Ezek segítségével lehetoség nyílt a fragmensek pályájának és ezzel sebességvektorainak, 
illetve az üveglapban eredetileg elfoglalt pozícióinak a meghatározására. A kísérlet nem mutatott korrelációt
a fragmensek mérete és sebessége között, azonban szignifikáns korrelációt sikerült kimutatni a fragmensek sebessége
és üveglapban elfoglalt eredeti pozíciója között. A probléma vizsgálata amiatt vált érdekessé, mert az eredmények
ellentmondanak egy 1991-ben A. Nakamura és A. Fujiwara által végzett kísérletnek. Ebben a kísérletben háromdimenziós
fragmentációt követtek nyomon a kamerákkal, azonban technikai okokból csak a nagyméretu fragmensek útját tudták
kello pontossággal nyomonkövetni. Meglepo módon határozott kapcsoltato találtak a fragmensek tömege és sebessége 
között. A sebesség a tömegnek csökkeno hatványfüggvényeként adódott, méghozzá egy univerzális 1/6-os hatványkitevovel

Képlet: $v~m^-(1/6)$

A munka során az impact-fragmentációt vizsgáljuk behatóbban
blabla konkuliok a végére....

Model Rendszer:
Kidolgoztunk egy kétdimenziós dinamikus modellt a deformálható, törheto szemcsés anyagok vizsgálatára. A modell segítségével
lehetoségünk nyílt törés és fragmentáció molekuláris dinamikai vizsgálatára különbözo kisérleti forgatókönyvek szerint.citeok
Az itt használt modell azon modellek kiterjesztése amelyekben a szemcsés anyagokat véletlenszeru alkú konvex poligonok alkotják.cite
Az anyag dinamiaki viselkedését az anyagot alkotó elemi részek (a deformálhatatlan polygonok) közti kölcsönhatások definiálásával írhatjuk
le. Két polygon között ható vonzó, összetartó erot rudakkal modellezük, amelyek megnyúlás vagy hajlás hatására el is törhetnek ezzel
megszüntetve a két polygon közötti összetartó erot. Azt hogy egy rúd eltört-e vagy sem, a megnyúlás illetve a meghajlás hatására ébredo feszültségek
és az ezek reltív fontosságát jellemezo $t_{\epsilon}$, $t_{\Theta}$ paraméterek segítségével dönthetjük el. 
A törések hatására a polygonok határfelületén repedések keletkeznek az anyag szerkezetében és törések sorozatának eredményeként az
anyag darabokra hullik. A fragmenseket olyan polygonok alkotják amelyek között a rudakkal modellezet összetartó kölcsönhatások még 
éppek.
!!!NA ide még kell kiegészíto info a következo részhez....

A rendszer idofejlodését az egyes polygonok mozgásegyenleteinek a megoldásával határozzuk meg. A szimuláció addig fut amíg a rendszer relaxálódik
ami esetünkben annyit jelent, hogy néhány száz egymást követo idolépésen keresztül nem történik újab törés a rendszerben, illetve a
polygonokat összeköto rudakban már nincsen tárolt energia... A modell további részletei megtalálhatók a bevezetoben.

Ahhoz hogy a kísérlet valósághu modellezését végrehajthassuk létrehoztuink egy L oldalhosszúságú négyzet alakú próbatestet. A lövedéket
a próbatest síkjával párhuzamosan lottük bele a testbe. Ahhoz hogy a kísérletben a lövéd megállítására alkalmazott tompító test hatását
modellezhessük, egy felületi polygonnak jelentos a test tömegközéppontja felé mutató kezdosebességet adtunk. Az xy ábrán a kezdo állapot
és néhány idolépés látható. 

ÁBRA + figcaption

Eredmények:

Nagy számú tesztet végeztünk, széles tartományon változtatva a lövedék kezdeti sebességét. A szimulációkban vizsgált rendszer mérete 
L=64 volt, azaz a rendszert megközelítoleg 3600 polygon alkotta. A lövedéket jelképezo polygont a próbatest jobb oldalának közepén választottuk 
ki. Lásd akármilyen ábra.
Annak érdekében, hogy a kapott eredmények statisztikai pontosságát növelhessük minden paraméter halmazt az test 240 különbözo realizációjára futattuk 
le. A modell rendszer pontos paraméter értékeit a 11,12 referencia cikkeiben találhatók. A végso fragmentálódott állapotban meghatároztuk a fragmensek 
méretét/tömegét, sebességét és az eredeti testben elfoglalt pozícióját.

A fragmensek térbeli sebességeloszlása:

Az xy ábrán a végso állapotról készült képen láthatjuk hogy a rendszer a lövés hatására a test táguláson ment keresztül minden fragmens a testbol 
kifelé repül. Már a kép alapján szemmel is megállapítható a fragmensek nem triviális térbeli tömeg és sebességeloszlása. Mint az látható a képen
a becsapódási terület környékén az anyag teljesen elpusztult, tulkajdonképpen minden polyhgon külön fragmenst alkot. Meglepo módon a polygonok egy
jelentos része a becsapódás sebességének írányával éppen ellenkezo irányú sebességgel rendelkezik, más szóval a becsapódási zóna közelében 
található részecskék vissza felé kilökodnek az anyagból.

ÁBRA + Caption

Az anyagban terjedo rugalmas hullámok interferenciájának eredményeként az anyag egy felületi rétege a lövedék pecsapódásával átellenes oldalon
szabályosan leszakad az anyagról, amelynek részei viszonylag nagy sebességgel távolodnak a próbatesttol. Az anyag belsejében található fragmensek
jóval nagyobbak és jelentosen alacsonyabb sebességgel rendelkeznek. Ahhoz hogy a szabad szemmel tapasztalható jelenségeket quantitatív módon is 
vizsgálhassuk meghatároztuk a fragmensek $v_x$ és $v_y$ sebességkomponenseinek átlagát az eredeti testben elfoglalt $x_0$, $y_0$ pozíciójuk függvényében.
Érdemes megjegyeznünk, hogy a test belsejében vannak olyan fragmensek amelyek gyakorlatilag az eredeti helyükön maradtak, azaz a végállapot beli
sebességük zérus. A lendületmegmaradás törvényének megfeleloen a $v_y(y_0)$ függvény szimmetrikus a $y_0=0$, figyelembe véve, hogy a koordináta rendszer 
origóját a lövedék becsapódásának pontjában vettük fel. A zérus sebességu fragmensek mindig az $y_0=0$ tengely mentén találhatók, az
$x_0$ pozíciójuk pedig a lövedék becsapódásának a sebességétol függ. Az xy ábrán jól látható, hogy a minkdét sebesség komponens a zérus sebességu
fragmensek helyétol távolodva monoton növekszik. A becsapódás hatására visszafelé kilökodött fragmensek illetve az ellenkezo oldalon leszakadó
fragmensek a becsapódási sebességnél nagyobb sebességeket is elérhetnek.

A fragmensek térbeli tömegeloszlása:

Az xy ábrából tisztán kiderül, hogy a fragmensek tömege erosen függ az eredeti testben elfoglalt pozíciójától. Kilejenthetjük, hogy a becsapódási
zóna környékén illetve a leszakadási zónában keletkezo fragmensek mérete nem összevetheto az eredeti test tömegével. A test belsejáben azonban nagyobb fragmensek
is képesek éppen maradni a végso állapotig. Az xy ábrán a fragmensek tömegét a fragmens eredeti testben elfoglalt helyének $x_0$ koordinátájának függvényében ábrázoltuk.
Az becsapódási zónában ahol az anyag teljes mértékben megsemmisült a fragmensek tömege nagyon kicsi, míg az anyag átellenes pontján a leszakadó réteg hatását egy kis 
csúcs egyértelmuen jelzi. Fontos észrevennünk, hogy a test belsejében a fragmensek tömegének eloszlása éles maximummal rendelkezik, ami a növekvo becsapódási
sebesség hatására eltolódik és egyre jobban kisimul. Az xy b ábrán jól látható, hogy egy megfeleloen megválasztott skálafüggvény segítségével
a a különbözo görbék szépen egymásra ejthetok. Az hogy a különbözo becsapódási sebességhez tartozó görbéket egymásra tudtuk ejteni egy egyszeru transzformációval
egy viszonylag egyszeru $m(x_0)$ alakot feltételez.

KÉPLET: $m(x_0) ~ v_{imp}^{-\alpha}f(x0/L+av_{imp})$,

ahol az $\alpha$ paraméter $\alpha=3.4\pm0.2$-nek adódott, f pedig egy skálafüggvény ahol az f argumentumában található $a$ eltolási tag az anyag belsejében keletkezett 
lökéshullám C sebességének a reciprokeként adódik. $a~=1/C$, ahol a hullám terjedéi sebessége $C=2600m/s$ ami független kisérletek eredményeként adódott. Az $xy$ függvény
alakjából következik, hogy a $m(x_0)$ függvény maximumának eltolódása arányos a becsapódó lövedék sebességével. Az xy ábra a tömeg eloszlást a fragmens eredeti testben
elfoglalt $y_0$ pozíciójának függvényében mutatja. Jól látható hogy az $m(y_0)$ függvény szimmetrikus az $y_0=0$ tengelyre. A szimmetriapont két oldalán egy-egy maximumot
figyelhetünk meg, amelyek pozíciója független a becsapódás sebességétol, így egy viszonylag egyszeru skálafüggvény adódik $m(y_0)$ esetén

\beq
m(y_0) ~ v_{imp}^{-\beta}g(y0/L).
\eeq

A kitevo $\beta=1.5\pm 0.1$-nek adódott, g pedik a skálafüggvény. A függvények egymásra esésének jó minosége jól látható az xyb ábrán.

Tömeg sebesség korreláció:

A fragmensek sebessége fontos a másodlagos fragmentációs rendszerek megértésében, amilyenek például a fragmensek másodlagos ütközései eredményeként létrejövő fragmentálódás
vagy az aszteroidák gravitációs tér hatása mellett történo ütközése eredményeként létrejövő fragmensek idofejlodése. A korábbi eredményekre alapozva, a modellünkbol
fontos következtetéseket vonhatunk le a fragmensek sebességére vonatkozóan. Mindezek mellett fontos megállapításokat tehetünk egyéb karakterisztikus jellemzok mint a tömeg
vagy a térbeli pozíció sebességfüggésérol.
Annak érdekében, hogy információt nyerhessünk a sebesség és tömeg lehetséges kapcsolatáról, meghatároztuk a fragmensek átlagos sebességét a tömegük függvényeként.
Jól látható az xy ábrán, hogy kis méretu fragmensek esetén majdnem két nagyságrenden keresztül a fragmens sebessége független a fragmen tömegétol. Azonban egy karakterisztikus
tömegértéknél azonban egy éles átmenetnek lehetünk tanui egy másik tartományba, ahol a sebesség hatványfüggvény szerint csökken a fragmens méretével.
A kapcsolatot tehát a következő függvénnyel jellemezhetjük:

\beq
v~a\cdot m^{-\gamma}
\eeq

A kitevo numerikusan $\gamma=1/3$-nak adódott. Az eredményeink megmagyarázzák a vizsgálatok motivációjául szolgáló kísérletek ellentmondásos eredményeit. A különbözo eredmények
az alkalmazott energia különbözoségébol származhatnak, hiszen egészen magas energia mellett a teljes anyag apró fragmensekre törik és mint azt láttuk egy kritikus tömeg alatt
nincs korreláció a sebesség és a tömeg között. Fontos megjegyezni, hogy az alkamazott energia mennyiségétol és így a tömeg-sebesség kapcsolat hiányától vagy meglététol függetlenül
a fragmensek tömegeloszlása folytonos függvény, méghozzá az irodlaomnak megfeleloen hatványfüggvény eloszlást követ, ami az xy ábrán is jól látható. Mindemellett nem csak
hogy a függvény alakja felel meg az irodalmi eredményeknek hanem a hatványfüggvény exponense az 1.87 értékkel számszeruen is egyezik a korábbi \verb=\cite= eredményekkel.


Összefoglalás:

Hat ez dobbenet jol ment...
